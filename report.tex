\documentclass[paper=a4, fontsize=11pt]{scrartcl}
\usepackage[T1]{fontenc}
\usepackage{fourier}

\usepackage[english]{babel}
\usepackage[protrusion=true,expansion=true]{microtype} 
\usepackage{amsmath,amsfonts,amsthm,amssymb}
\usepackage[pdftex]{graphicx} 
\usepackage{url}

\usepackage{tikz} 
\usepackage{tikzscale}

\usepackage{parskip}

\usepackage{hyperref}
\hypersetup{colorlinks=true, citecolor=black, linkcolor=black, urlcolor=blue}
\urlstyle{same}

\usepackage{subfigure}

\usepackage{caption}

\usepackage[outputdir=out]{minted}
\usemintedstyle{tango}

\newtheorem{proposition}{Proposition}

\usepackage{sectsty}
\allsectionsfont{\centering \normalfont\scshape}
\usepackage{fancyhdr}
\pagestyle{fancyplain}
\fancyhead{}
\fancyfoot[L]{}
\fancyfoot[C]{}
\fancyfoot[R]{\thepage}
\renewcommand{\headrulewidth}{0pt}
\renewcommand{\footrulewidth}{0pt}
\setlength{\headheight}{13.6pt}

\newcommand{\horrule}[1]{\rule{\linewidth}{#1}}

\title{
    \usefont{OT1}{bch}{b}{n}
    \normalfont \normalsize \textsc{Università degli Studi di Verona} \\ [15pt]
    Master's degree in computer science and engineering / Big Data \\ [15pt]
    \horrule{0.5pt} \\[0.4cm]
    \huge Counting the number of triangles in a very large graph using the TTP algorithm \\
    \horrule{2pt} \\[0.5cm]
}
\author{
    \normalfont \normalsize
    Elia Perantoni, VR472361
    \\[-3pt] \normalsize \today
}
\date{}

\begin{document}
\maketitle
\section{Introduction}
A triangle $\Delta(u, v, x)$ in a undirected graph $G=(V, E)$ is a triple of
nodes $u, v, x$ such that all edges between them are in $E$. That is: $\{(u,v),
(u,x), (v, x)\}\subseteq E$.

\begin{figure}[htb]
    \centering
    \begin{tikzpicture}[node distance={15mm}, thick, every node/.style={draw, circle, minimum size=22pt}]
        \node (u) {$u$}; 
        \node (v) [below left of=u] {$v$}; 
        \node (x) [below right of=u] {$x$};
        \draw (u) -- (v) -- (x) -- (u);
        \node (a) [left of=v] {$a$};
        \node (b) [below left of=a] {$b$};
        \node (c) [below right of=x] {$c$};
        \draw (v) -- (a) -- (b);
        \draw (x) -- (c);
    \end{tikzpicture}
    \caption{A triangle $\Delta(u, v, x)$ inside a larger graph}
\end{figure}

The problem of counting triangles in a graph inherits great importance from
several interesting metrics on graphs that build upon it. The clustering
coefficient, for instance, is a measure of the degree to which nodes in a graph
tend to cluster together. Evidence suggests that real world graphs, social
networks in particular, tend to create tightly knit groups characterized by a
relatively high density of ties. This metric, arguably a very insightful one,
requires the number of triangles to be known, in order to compute it. Hence, its
relevance.

In recent years, internal and external memory algorithms have been developed to
solve the problem of triangle counting; but some of the graphs that we wish to
analyze have become so large that they can no longer fit inside a single
machine.

A popular technique for distributed computation is \emph{Map-Reduce}: a
\emph{map} function is applied to each element in the input collection
individually to produce key-value pairs, the shuffle phase then merges all
values belonging to the same key and, finally, a \emph{reduce} function is
applied to each group of values (one for each key) to produce the final result.
What makes Map-Reduce so powerful is the fact that data can be split in
partitions to which the map and reduce functions can be applied independently.
With this mechanism, processing enormous quantities of data becomes possible as
the work can be distributed among a large number of machines.

This report comes with, and provides context to, a Jupyter notebook in which a
recently developed Map-Reduce algorithm known as \emph{TTP (Triangle Type
Partition)} \cite{park2013efficient} is implemented, tested, analyzed and
compared with one of its predecessor: \emph{GP (Graph Partition)}
\cite{suri2011counting}.

We will first be introducing and discussing GP before moving to TTP because the
two algorithms have many concepts in common but GP is easier to understand.

\section{The tools}
Arguably, the most known Map-Reduce implementation is \emph{Apache Hadoop},
which also provides a distributed file system for data persistence: \emph{HDFS
(Hadoop Distributed File System)}. However, using Hadoop is somewhat tedious as
the provided functionality is limited to plain Map-Reduce only. Also, HDFS comes
with sup-optimal performance as secondary storage is used.

\emph{Apache Spark} is a software based on Hadoop that provides the framework
and unified set of APIs for performing a wide range of distributed computation
operations that go beyond just simple Map-Reduce. Spark also improves on HDFS by
using what's known as: \emph{RDD (Resilient Distributed Dataset)}. It's a
managed data structure that gets sharded and distributed (possibly with
replication) among the machines belonging to the same cluster. What's different
from HDFS is that RDDs' shards are stored in internal memory, thus noticeably
improving the performance of read/write operations.

Our implementation is Apache Spark based.

\section{Graph Partition algorithm}
In GP \cite{suri2011counting}, the input graph $G=(V, E)$ is split into $\rho$
partitions ($G_0, G_1, \dots, G_{\rho-1}$) using a partitioning function $P$
such that $\forall u \in V \ldotp P(u) \in [0, \rho-1]$. This gives
$V=\bigcup_{i=0}^{\rho-1} V_i$ and $\forall_{i\neq j} \ldotp V_i \cap V_j =
\emptyset$.

3 partitions $G_i$, $G_j$ and $G_k$ can be combined into one 3-partition
$G_{ijk}$ by taking the union of the nodes $V_{ijk} \triangleq V_i \cup V_j \cup
V_k$ and inducing the edges from the original $G$. See fig. \ref{some3parts} for
some examples.

\begin{figure}[h]
    \centering
    \begin{tikzpicture}[main/.style={draw, circle, minimum size=22pt, inner sep=0pt}]
        \node[main] (1) at (2, 7) {1};
        \node[main] (2) at (0, 6) {2};
        \node[main] (3) at (2, 5) {3};

        \node[main] (4) at (5, 7) {4};
        \node[main] (5) at (7, 6) {5};
        \node[main] (6) at (5, 5) {6};

        \node[main] (7) at (1, 2) {7};
        \node[main] (8) at (0, 0) {8};
        \node[main] (9) at (2, 1) {9};

        \node[main] (10) at (5, 1) {10};
        \node[main] (11) at (6, 2) {11};
        \node[main] (12) at (7, 0) {12};

        \draw (1) -- (2) -- (3) -- (1);
        \draw (4) -- (5) -- (6) -- (4);
        \draw (8) -- (7) -- (9);
        \draw (10) -- (12) -- (11);

        \draw (4) -- (3) -- (6);
        \draw (7) -- (10) -- (6) -- (7);
        \draw (3) -- (7);

        \draw[dashed] (-1, 3.5) -- (8, 3.5);
        \draw[dashed] (3.5, -1) -- (3.5, 8);

        \node at (-1.5,5.25) {Partition 0};
        \node at (8.5,5.25) {Partition 1};
        \node at (-1.5,1.75) {Partition 2};
        \node at (8.5,1.75) {Partition 3};
    \end{tikzpicture}
    \caption{A simple graph divided into 4 partitions}
    \label{graph}
\end{figure}

\begin{figure}[h]
    \centering
    \subfigure[$G_{012}$]{
        \includegraphics[width=0.3\linewidth]{threeparts1}
    }
    \hspace{40pt}
    \subfigure[$G_{123}$]{
        \includegraphics[width=0.3\linewidth]{threeparts2}
    }
    \caption{Some 3-partitions from the graph in fig. \ref{graph}}
    \label{some3parts}
\end{figure}

The algorithm finds the triangles in every 3-partition $G_{ijk}$ with $0 \le i <
j < k \le \rho - 1$ and assigns a weight to each. The sum of weights across all
partitions will give the correct number of triangles in $G$.

Some triangles are seen in more than one 3-partition. Take a triangle $\Delta(u,
v, x)$ entirely contained within partition $G_0=(V_0, E_0)$ for instance, it
will be observed repeatedly in any 3-partition $G_{0jk}$ with $j,k\neq 0$. The
purpose of the weight system then is to counteract this effect, ensuring that
the sum of all weights emitted for a triangle is always $1$. 

The weight to emit for a triangle solely depends on the number of partitions
that its nodes span:
\[
    w(\Delta(u,v,x)) = \begin{cases}
        \frac{1}{\binom{\rho-1}{2}} & P(u) = P(v) = P(x)  \\
        \frac{1}{\rho-2} & P(u) = P(v) \vee P(v) = P(x) \vee P(u) = P(x) \\
        1 & \text{otherwise}
    \end{cases}
\]

\begin{proposition}
    The sum of weights emitted by GP for any given triangle $\Delta(u, v, x)$ is $1$.
\end{proposition}
\begin{proof}
    Trivially, exactly one of these must be true:
    \begin{enumerate}
        \item $\Delta(u, v, x)$ is entirely contained within a single partition $G_A$
        \item $\Delta(u, v, x)$ is entirely contained within two partitions $G_A$ and $G_B$
        \item $\Delta(u, v, x)$ spans three partitions $G_A$, $G_B$ and $G_C$
        with each node belonging to a different one
    \end{enumerate}
    Therefore, we proceed with a proof by cases:
    \begin{enumerate}
        \item $\Delta(u, v, x)$ appears in any 3-partition $G_{ijk}$ where
        either $i$, $j$ or $k$ is equal to $A$. Therefore, two variables remain
        free and their values are chosen between $\rho-1$ possible partitions
        (all except A). Hence, $\Delta(u, v, x)$ appears in $\binom{\rho-1}{2}$
        partitions. For each of those a weight equal to
        $\frac{1}{\binom{\rho-1}{2}}$ is emitted, giving
        $\binom{\rho-1}{2}*\frac{1}{\binom{\rho-1}{2}}=1$.

        \item $\Delta(u, v, x)$ appears in any 3-partition $G_{ijk}$ with $\{i,
        j, k\} \subseteq \{A, B\}$. This means that one variable remains free
        and its value must be chosen among $\rho-2$ possible partitions (all
        except $A$ and $B$). Hence, $\Delta(u, v, x)$ appears in $\rho-2$
        partitions. For each of those a weight equal to $\frac{1}{\rho-2}$ is
        emitted, giving $(\rho-2)*\frac{1}{\rho-2}=1$.

        \item $\Delta(u, v, x)$ appears just in $G_{ABC}$. The weight $1$ will
        be emitted just once, trivially making the sum equal to $1$.
    \end{enumerate}
\end{proof}

The \emph{map} step consists of emitting an edge for every 3-partition that it
belongs to. The \emph{reduce} step is given a complete 3-partition and its job
is that of enumerating the triangles and emitting a weight for each. A
subsequent Map-Reduce pass is required to sum up the weights, giving the final
result.

The distributed nature of GP emerges when observing that the actual triangle
counting process can be performed on every 3-partition independently, using any
internal memory algorithm.

Depending on the size of $G$, a sufficiently large number of partitions $\rho$
will make sure that each is sufficiently small to fit inside a single machine.

The main issues with GP is that edges are emitted many times over, to account
for all 3-partitions in which they appear, and triangles are processed
redundantly. Take a triangle entirely contained within a single partition for
instance, with $\rho=10$. It will be processed in $\binom{\rho-1}{2}=36$
different 3-partitions. The improved algorithm, TTP, brings that number down to
just 9.

\section{Triangle Type Partition algorithm}

The TTP algorithm \cite{park2013efficient} improves on GP by reducing the amount
of redundancy and thus, the amount of data transferred in the cluster.

First, we formally classify triangles much in the same way as we did before for
our proof:
\begin{description}
    \item[Type I] Is a triangle $\Delta(u, v, x)$ spanning 1 partition
    \item[Type II] Is a triangle $\Delta(u, v, x)$ spanning 2 partitions
    \item[Type III] Is a triangle $\Delta(u, v, x)$ spanning 3 partitions
\end{description}

For instance, in fig. \ref{graph} $\Delta(1,2,3)$ is a Type I triangle,
$\Delta(3,4,6)$ is a Type II triangle and $\Delta(3,6,7)$ is a Type III
triangle.

Then, edges are also classified as being either \emph{inner} or \emph{outer}. An
edge $(u,v)$ is said to be inner when $u$ and $v$ belong to the same partition.
Otherwise, it's an outer edge.

A 3$^\prime$-partition (note the prime), denoted $G^{\prime}_{ijk}$, is a
3-partition composed only of outer edges. TTP reduces redundancy by searching
for Type III triangles in 3$^\prime$-partitions while and Type I and II
triangles in 2-partitions.

See fig. \ref{ttp} for some examples of 2-partitions and 3$^\prime$-partitions.

\begin{figure}[h]
    \centering

    \subfigure[$G_{01}$]{
        \includegraphics[width=0.3\linewidth]{twoparts1.tikz}
    }
    \hspace{20pt}
    \subfigure[$G_{02}$]{
        \includegraphics[height=0.3\linewidth]{twoparts2.tikz}
    }
    \hspace{20pt}
    \subfigure[$G_{23}$]{
        \includegraphics[width=0.3\linewidth]{twoparts3.tikz}
    }

    \par\bigskip

    \subfigure[$G^\prime_{012}$]{
        \includegraphics[width=0.3\linewidth]{threepartsprime1}
    }
    \hspace{20pt}
    \subfigure[$G^\prime_{123}$]{
        \includegraphics[width=0.3\linewidth]{threepartsprime2}
    }
    \caption{Some examples of 2-partitions and 3$^\prime$-partitions from the
    graph in fig. \ref{graph}}
    \label{ttp}
\end{figure}

The interesting observation to make here is that Type III triangles are entirely
made up of outer edges and can therefore be correctly counted in
3$^\prime$-partitions that lack inner edges; thus making them smaller and
quicker to process. Additionally, Type I and Type II have at least one inner
edge, so they are not repeatedly processed in 3$^\prime$-partitions simply
because they don't exist there. The number of 3$^\prime$-partitions partitions
grows exponentially with $\rho$, so stripping them of inner edges and only
processing Type III triangles there makes for a big performance improvement.

The overall structure of the TTP algorithm is similar to that of GP: the
\emph{map} step emits a pair $(p, e)$ for every edge $e$ that belongs to
partition $p$, then the \emph{reduce} step receives an entire partition (that
might be a 2-partition or a 3$^\prime$-partition) and emits a weight for each
triangle found there. The sum of all weights emitted gives the total number of
triangles in the initial graph $G$.

In TTP, the weights are as follows:
\[
    w(\Delta(u,v,x)) = \begin{cases}
        \frac{1}{\rho-1} & P(u) = P(v) = P(x)  \\
        1 & \text{otherwise}
    \end{cases}
\]
So Type I triangles get $\frac{1}{\rho-1}$ while Type II and Type III triangles
get $1$.

\begin{proposition}
    The sum of weights emitted by TTP for any given triangle $\Delta(u, v, x)$ is $1$.
\end{proposition}
\begin{proof}
    Every triangle is either Type I, Type II or Type III. Let's proceed with a
    proof by cases.
    \begin{description}
        \item[Type I] Suppose $\Delta(u, v, x)$ is entirely contained within
        partition $G_A$. Then, it appears in all 2-partitions $G_{ij}$ with
        $i=A$ or $j=A$. It cannot appear in 3$^\prime$-partitions because a Type
        I triangle is entirely made out of inner edges, which
        3$^\prime$-partitions lack. Therefore, there are $\rho-1$ partitions in
        which the triangle appears. And every time that happens a weight equal
        to $\frac{1}{\rho-1}$ is emitted. Giving $(\rho-1)*\frac{1}{\rho-1}=1$.

        \item[Type II] Suppose $\Delta(u, v, x)$ is entirely contained within
        partition $G_A$ and $G_B$ (suppose also wlog that $A<B$). Then, it
        appears only in 2-partition $G_{AB}$. It cannot appear in
        3$^\prime$-partitions because a Type II triangle has one inner edge,
        which 3$^\prime$-partitions lack. Therefore, there is just one partition
        in which the triangle appears. Since $1$ is emitted just once, the sum
        is trivially equal to $1$.

        \item[Type III] Suppose $\Delta(u, v, x)$ spans three partitions $G_A$,
        $G_B$ and $G_C$ (suppose also wlog that $A<B<C$). Then, it appears only
        in 3$^\prime$-partition $G_{ABC}$. It cannot appear in 2-partitions
        simply because the triangle is composed of nodes from three distinct
        partitions. Since $1$ is emitted just once, the sum is trivially equal
        to $1$.
    \end{description}
\end{proof}

\section{Datasets}
For this project, undirected graphs from the
\href{http://snap.stanford.edu/data/index.html}{Standford Large Network Dataset
Collection} were used. In particular, the software is tested against these (in
order of disk size):
\begin{enumerate}
    \item ego-Facebook (1612010 triangles)
    \item email-Enron (727044 triangles)
    \item com-Amazon (667129 triangles)
    \item roadNet-CA (120676 triangles)
\end{enumerate}

To make the whole algorithm easier to understand and reason about, the same
example from \cite{park2013efficient} is also used before getting to more
complicated datasets.

Each dataset is given as a text file in the \mintinline{text}{datasets/}
directory. Nodes are identified by means of a unique integer and for each edge
$(u,v) \in E$ there is a line in the file with the node ids of $u$ and $v$,
separated by whitespace. \emph{Edges are not reported twice}, meaning there is
no \mintinline{text}{"v u"} line when \mintinline{text}{"u v"} is already there.

\section{Implementation in depth}
In this section, we are going to walk through the implementation provided along
with this report as a Jupyter notebook. The same example graph used in
\cite{park2013efficient} will be used throughout.

\section{Impact of $\rho$}

\section{Performance comparison TTP vs GP}

\bibliographystyle{plain}
\bibliography{refs}
\end{document}
