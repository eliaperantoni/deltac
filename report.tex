\documentclass[paper=a4, fontsize=11pt]{scrartcl}
\usepackage[T1]{fontenc}
\usepackage{fourier}

\usepackage[english]{babel}
\usepackage[protrusion=true,expansion=true]{microtype} 
\usepackage{amsmath,amsfonts,amsthm}
\usepackage[pdftex]{graphicx} 
\usepackage{url}

\usepackage{tikz} 

\usepackage{parskip}

\usepackage[outputdir=out]{minted}
\usemintedstyle{tango}

\newtheorem{proposition}{Proposition}

\usepackage{sectsty}
\allsectionsfont{\centering \normalfont\scshape}
\usepackage{fancyhdr}
\pagestyle{fancyplain}
\fancyhead{}
\fancyfoot[L]{}
\fancyfoot[C]{}
\fancyfoot[R]{\thepage}
\renewcommand{\headrulewidth}{0pt}
\renewcommand{\footrulewidth}{0pt}
\setlength{\headheight}{13.6pt}

\numberwithin{equation}{section}
\numberwithin{figure}{section}
\numberwithin{table}{section}

\newcommand{\horrule}[1]{\rule{\linewidth}{#1}}

\title{
    \usefont{OT1}{bch}{b}{n}
    \normalfont \normalsize \textsc{Università degli Studi di Verona} \\ [15pt]
    Master's degree in computer science and engineering / Big Data \\ [15pt]
    \horrule{0.5pt} \\[0.4cm]
    \huge Counting the number of triangles in a very large graph using the TTP algorithm \\
    \horrule{2pt} \\[0.5cm]
}
\author{
    \normalfont \normalsize
    Elia Perantoni, VR472361
    \\[-3pt] \normalsize \today
}
\date{}

\begin{document}
\maketitle
\section{Introduction}
A triangle $\Delta(u, v, x)$ in a undirected graph $G=(V, E)$ is a triple of
nodes $u, v, x$ such that all edges between them are in $E$. That is: $\{(u,v),
(u,x), (v, x)\}\subseteq E$.

\begin{figure}[htb]
    \centering
    \begin{tikzpicture}[node distance={15mm}, thick, every node/.style={draw, circle, minimum size=22pt}]
        \node (u) {$u$}; 
        \node (v) [below left of=u] {$v$}; 
        \node (x) [below right of=u] {$x$};
        \draw (u) -- (v) -- (x) -- (u);
        \node (a) [left of=v] {$a$};
        \node (b) [below left of=a] {$b$};
        \node (c) [below right of=x] {$c$};
        \draw (v) -- (a) -- (b);
        \draw (x) -- (c);
    \end{tikzpicture}
    \caption{A triangle $\Delta(u, v, x)$ inside a larger graph}
\end{figure}

The problem of counting triangles in a graph inherits great importance from
several interesting metrics on graphs that build upon it. The clustering
coefficient, for instance, is a measure of the degree to which nodes in a graph
tend to cluster together. Evidence suggests that real world graphs, social
networks in particular, tend to create tightly knit groups characterised by a
relatively high density of ties. This metric, arguably a very insightful one,
requires the number of triangles to be known, in order to compute it. Hence, its
relevance.

In recent years, internal and external memory algorithms have been developed to
solve the problem of triangle counting; but some of the graphs that we wish to
analyse have become so large that they can no longer fit inside a single
machine.

A popular technique for distributed computation is \emph{Map-Reduce}: a
\emph{map} function is applied to each element in the input collection
individually to produce key-value pairs, the shuffle phase then merges all
values belonging to the same key and, finally, a \emph{reduce} function is
applied to each group of values (one for each key) to produce the final result.
What makes Map-Reduce so powerful is the fact that data can be split in
partitions to which the map and reduce functions can be applied independently.
With this mechanism, processing enormous quantities of data becomes possible as
the work can be distributed among a large number of machines.

This report comes with, and provides context to, a Jupyter notebook in which a
recently developed Map-Reduce algorithm known as \emph{TTP (Triangle Type
Partition)} is implemented, tested, analysed and compared with one of its
predecessor: \emph{GP (Graph Partition)}.

We will first be introducing and discussing GP before moving to TTP because the
two algorithms have many concepts in common but GP is easier to understand.

\section{The tools}
Arguably, the most known Map-Reduce implementation is \emph{Apache Hadoop},
which also provides a distributed file system for data persistence: \emph{HDFS
(Hadoop Distribtued File System)}. However, using Hadoop is somewhat tedious as
the provided functionality is limited to plain Map-Reduce only. Also, HDFS comes
with sup-optimal performance as secondary storage is used.

\emph{Apache Spark} is a software based on Hadoop that provides the framework
and unified set of APIs for peforming a wide range of distributed computation
operations that go beyond just simple Map-Reduce. Spark also improves on HDFS by
using what's known as: \emph{RDD (Resilient Distributed Dataset)}. It's a
managed data structure that gets sharded and distributed (possibly with
replication) among the machines belonging to the same cluster. What's different
from HDFS is that RDDs' shards are stored in internal memory, thus noticeably
improving the performance of read/write operations.

Our implementation is Apache Spark based.

\section{Graph Partition algorithm}
In GP, the input graph $G=(V, E)$ is split into $\rho$ partitions ($G_0, G_1,
\dots, G_{\rho-1}$) using a partitioning function $P$ such that $\forall u \in V
\ldotp P(u) \in [0, \rho-1]$. This gives $V=\bigcup_{i=0}^{\rho-1} V_i$
and $\forall_{i\neq j} \ldotp V_i \cap V_j = \emptyset$.

Then, the algorithm finds the triangles in every 3-partition $G_{ijk}$ with $0
\le i < j < k \le \rho - 1$ and assigns a weight to each. The sum of weights
across all partitions will give the correct number of triangles in $G$.

Because some triangles are seen in more than one 3-partition, the purpose of the
weight system is to counteract this effect; ensuring that the weights emitted
for a given triangle always sum up to $1$. Take a triangle $\Delta(u, v, x)$
entirely contained within partition $G_0=(V_0, E_0)$ for instance, it will be
observed repeatedly in any 3-partition $G_{0jk}$ with $j,k\neq 0$.

The correct weight for a triangle can be found just by looking at the partitions
that its nodes belong to:
\[
    w(\Delta(u,v,x)) = \begin{cases}
        \frac{1}{\binom{\rho-1}{2}} & P(u) = P(v) = P(x)  \\
        \frac{1}{\rho-2} & P(u) = P(v) \vee P(v) = P(x) \vee P(u) = P(x) \\
        1 & \text{otherwise}
    \end{cases}
\]

\begin{proposition}
    The sum of weights emitted by GP for any given triangle $\Delta(u, v, x)$ is $1$.
\end{proposition}
\begin{proof}
    Trivially, exactly one of these must be true:
    \begin{enumerate}
        \item $\Delta(u, v, x)$ is entirely contained within a single partition $G_A$
        \item $\Delta(u, v, x)$ is entirely contained within two partitions $G_A$ and $G_B$
        \item $\Delta(u, v, x)$ spans three partitions $G_A$, $G_B$ and $G_C$
        with each node belonging to a different one
    \end{enumerate}
    Therefore, we proceed with a proof by cases:
    \begin{enumerate}
        \item $\Delta(u, v, x)$ appears in any 3-partition $G_{ijk}$ where
        either $i$, $j$ or $k$ is equal to $A$. Therefore, two variables remain
        free and their values are chosen between $\rho-1$ possible partitions
        (all except A). Hence: $\binom{\rho-1}{2}$. Because the triangle appears
        in $\binom{\rho-1}{2}$ 3-partitions, and the weight is emitted once for
        each, it must be equal to $\frac{1}{\binom{\rho-1}{2}}$ to ensure that
        $\binom{\rho-1}{2}* \frac{1}{\binom{\rho-1}{2}} = 1$.

        \item $\Delta(u, v, x)$ appears in any 3-partition $G_{ijk}$ with $\{i,
        j, k\} \subseteq \{A, B\}$. This means that one variable remains free
        and its value must be choosen among $\rho-2$ possible partitions (all
        except A and B). Because the triangle appears in ($\rho-2$)
        3-partitions, and the weight is emitted once for each, it must be equal
        to $\frac{1}{\rho-2}$ to ensure that $(\rho-2)*\frac{1}{\rho-2}=1$.

        \item $\Delta(u, v, x)$ appears just in $G_{ABC}$. The weight $1$ will
        be emitted just once, trivially making the sum equal to $1$.
    \end{enumerate}
\end{proof}

The distributed nature of GP emerges when observing that each 3-partition can be
processed independently using an internal memory algorithm. This step acts
as a \emph{map} function, while summing the weights can be thought of as a
\emph{reduce} step.

Depending on the size of $G$, a sufficiently large number of partitions ($\rho$)
will make sure that each is sufficiently small to fit inside a single machine.

\section{Triangle Type Partition algorithm}

\section{Implementation explained}

\section{Impact of $\rho$}

\section{Performance comparision TTP vs GP}
\end{document}
